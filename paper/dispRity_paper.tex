\documentclass[12pt,letterpaper]{article}
\usepackage{natbib}

%Packages
\usepackage{pdflscape}
\usepackage{fixltx2e}
\usepackage{textcomp}
\usepackage{fullpage}
\usepackage{float}
\usepackage{latexsym}
\usepackage{url}
\usepackage{epsfig}
\usepackage{graphicx}
\usepackage{amssymb}
\usepackage{amsmath}
\usepackage{bm}
\usepackage{array}
\usepackage[version=3]{mhchem}
\usepackage{ifthen}
\usepackage{caption}
\usepackage{hyperref}
\usepackage{amsthm}
\usepackage{amstext}
\usepackage{enumerate}
\usepackage[osf]{mathpazo}
\usepackage{dcolumn}
\usepackage{lineno}
\usepackage{dcolumn}
\newcolumntype{d}[1]{D{.}{.}{#1}}

\pagenumbering{arabic}


%Pagination style and stuff
\linespread{2}
\raggedright
\setlength{\parindent}{0.5in}
\setcounter{secnumdepth}{0} 
\renewcommand{\section}[1]{%
\bigskip
\begin{center}
\begin{Large}
\normalfont\scshape #1
\medskip
\end{Large}
\end{center}}
\renewcommand{\subsection}[1]{%
\bigskip
\begin{center}
\begin{large}
\normalfont\itshape #1
\end{large}
\end{center}}
\renewcommand{\subsubsection}[1]{%
\vspace{2ex}
\noindent
\textit{#1.}---}
\renewcommand{\tableofcontents}{}
%\bibpunct{(}{)}{;}{a}{}{,}

%---------------------------------------------
%
%       START
%
%---------------------------------------------

\begin{document}

%Running head
\begin{flushright}
Version dated: \today
\end{flushright}
\bigskip
\noindent RH: dispRity package.

\bigskip
\medskip
\begin{center}

\noindent{\Large \bf \texttt{dispRity}: a modular \texttt{R} package for measuring disparity.} 
\bigskip

\noindent {\normalsize \sc Thomas Guillerme$^1$$^,$$^*$}\\
\noindent {\small \it 
$^1$Imperial College London, Silwood Park Campus, Department of Life Sciences, Buckhurst Road, Ascot SL5 7PY, United Kingdom.\\}
\end{center}
\medskip
\noindent{*\bf Corresponding author.} \textit{guillert@tcd.ie}\\  
\vspace{1in}

%Line numbering
\modulolinenumbers[1]
\linenumbers

%---------------------------------------------
%
%       ABSTRACT
%
%---------------------------------------------

\newpage
\begin{abstract}

\begin{enumerate}
    \item I present \texttt{dispRity}, an \texttt{R} package form measuring disparity from ordinated matrices. 
    \item Disparity designates a suit of metrics to describe an ordinated matrix (also referred as morpho/eco/niche/etc.-space).
    \item The package is based on a highly modular architecture were metrics for measuring disparity and tests for testing evolutionary hypothesis are provided as standalone functions.
    \item This modular architecture allows great plasticity in using this package, namely by letting users define their own metric of disparity.
    \item The package also provides numerous tools for modifying, summarising and plotting \texttt{dispRity} objects.
\end{enumerate}

\end{abstract}

\noindent (Keywords: disparity, ordination, multidimensionality, disparity through time, palaeobiology, ecology)\\

\vspace{1.5in}

\newpage 

%---------------------------------------------
%
%       INTRODUCTION
%
%---------------------------------------------

\section{Introduction}
Disparity = multidimensionality

Disparity can be measured in many ways in palaeobiology (cite many examples) or in ecology (donohue and stuff).

Although some packages have disparity components (Claddis, geomorph, adonis), all have their own definition of disparity and makes the whole work idiosuncratic

Therefore I made this package to help measuring disparity

\section{Description}

\subsubsection{The package's data}

Beck and Lee data
Beck and Lee tree
Beck and Lee ages

disparity  disparity data

\subsubsection{Running example}


\subsection{The \texttt{dispRity} package}

\subsubsection{The package's main functions}

\begin{table}
    \begin{tabular}{ll}
        \hline
        Function & Description \\ 
        \hline
        \texttt{geomorph.ordination} & Imports data from the \texttt{geomorph} package. \\
        \texttt{Claddis.ordination} & Imports data from the \texttt{Claddis} package. \\
        \texttt{custom.subsamples} & Separates ordinated data in custom subsamples. \\
        \texttt{time.subsamples} & Separates ordinated data in time subsamples. \\
        \texttt{boot.matrix} & Bootstraps and rarefies ordinated data or a \texttt{dispRity} object. \\
        \texttt{dispRity} & Calculates disparity from an ordinated matrix or a \texttt{dispRity} object. \\
        \texttt{summary.dispRity} & Summarises a \texttt{dispRity} object. \\
        \texttt{plot.dispRity} & Plots a \texttt{dispRity} object. \\
        \texttt{test.dispRity} & Applies statistical tests to a \texttt{dispRity} object.\\
        \texttt{dispRity.per.group} & Pipelined disparity per groups analysis. \\
        \texttt{dispRity.through.time} & Pipelined disparity through time analysis. \\
        \texttt{dispRity.metrics} & A set of implemented disparity metrics functions.\\
        \texttt{dispRity.utilities} & A set of functions for modifying a \texttt{dispRity} object.\\
        \hline
    \end{tabular}
    \caption{Main functions in the \texttt{dispRity} package.}
\end{table}

\subsection{Fork flow}

Workflow figure

\subsubsection{Input data}

\subsubsection{Manipulating the data}

\subsubsection{Measuring disparity}

\subsubsection{Summarising the data}

Link to graph one two three

\subsubsection{Testing hypothesis}

\subsection{The \texttt{dispRity} object}

Disparity object structure figure

\subsection{Additional functions and data}

The package also contains additional functions not directly linked to the disparity workflow nor to the \texttt{dispRity} object but that can be used as utilities for the studying discrete morphological data.

\begin{table}
    \begin{tabular}{ll}
        \hline
        Function & Description \\ 
        \hline
        \texttt{char.diff} & Calculates character differences. \\
        \texttt{clean.data} & Cleans phylogenetic data (tree and table). \\
        \texttt{sim.morpho} & Simulates morphological discrete data. \\
        \texttt{slice.tree} & Slice through a phylogenetic tree. \\
        \texttt{space.maker} & Creates multidimensional spaces. \\
        \texttt{tree.age} & Calculates the age of nodes and tips in a tree. \\
        \hline
    \end{tabular}
    \caption{Additional functions in the \texttt{dispRity} package.}
\end{table}

Additionally, this package also contains an ecology type data set from McClean

McClean data

\section{Conclusion}

This package is good.

\section{Data availability and reproducibility}
Data are available on Dryad or Figshare.
Code for reproducing the analyses is available on GitHub (\url{github.com/TGuillerme/SpatioTemporal_Disparity}).

\section{Acknowledgments}
Thanks to Dave Bapst, Martin Brazeau, Natalie Cooper, Andrew Jackson, Graeme Lloyd and Emma Sherratt.
I acknowledge support from European Research Council under the European Union's Seventh Framework Programme (FP/2007 – 2013)/ERC Grant Agreement number 311092 awarded to Martin D. Brazeau.

\section{Funding}
This work was funded by the European Research Council under the European Union's Seventh Framework Programme (FP/2007–2013)/ERC Grant Agreement number 311092.

\bibliographystyle{sysbio}
\bibliography{References}

\end{document}
