\documentclass[12pt,letterpaper]{article}
\usepackage{natbib}

%Packages
\usepackage{pdflscape}
\usepackage{fixltx2e}
\usepackage{textcomp}
\usepackage{fullpage}
\usepackage{float}
\usepackage{latexsym}
\usepackage{url}
\usepackage{epsfig}
\usepackage{graphicx}
\usepackage{amssymb}
\usepackage{amsmath}
\usepackage{bm}
\usepackage{array}
\usepackage[version=3]{mhchem}
\usepackage{ifthen}
\usepackage{caption}
\usepackage{hyperref}
\usepackage{amsthm}
\usepackage{amstext}
\usepackage{enumerate}
\usepackage[osf]{mathpazo}
\usepackage{dcolumn}
\usepackage{lineno}
\usepackage{dcolumn}
\newcolumntype{d}[1]{D{.}{.}{#1}}

\pagenumbering{arabic}


%Pagination style and stuff
\linespread{2}
\raggedright
\setlength{\parindent}{0.5in}
\setcounter{secnumdepth}{0} 
\renewcommand{\section}[1]{%
\bigskip
\begin{center}
\begin{Large}
\normalfont\scshape #1
\medskip
\end{Large}
\end{center}}
\renewcommand{\subsection}[1]{%
\bigskip
\begin{center}
\begin{large}
\normalfont\itshape #1
\end{large}
\end{center}}
\renewcommand{\subsubsection}[1]{%
\vspace{2ex}
\noindent
\textit{#1.}---}
\renewcommand{\tableofcontents}{}
%\bibpunct{(}{)}{;}{a}{}{,}

%---------------------------------------------
%
%       START
%
%---------------------------------------------

\begin{document}

%Running head
\begin{flushright}
Version dated: \today
\end{flushright}
\bigskip
\noindent RH: dispRity package.

\bigskip
\medskip
\begin{center}

\noindent{\Large \bf \texttt{dispRity}: a modular \texttt{R} package for measuring disparity.} 
\bigskip

\noindent {\normalsize \sc Thomas Guillerme$^1$$^,$$^*$, Mark Puttick$^2$, Natalie Cooper$^3$}\\
\noindent {\small \it 
$^1$Imperial College London, Silwood Park Campus, Department of Life Sciences, Buckhurst Road, Ascot SL5 7PY, United Kingdom.\\}
\end{center}
\medskip
\noindent{*\bf Corresponding author.} \textit{guillert@tcd.ie}\\  
\vspace{1in}

%Line numbering
\modulolinenumbers[1]
\linenumbers

%---------------------------------------------
%
%       ABSTRACT
%
%---------------------------------------------

\newpage
\begin{abstract}

\begin{enumerate}
    \item We present \texttt{dispRity}, an \texttt{R} package in multidimensional spaces. 
    \item Disparity designates a suit of metrics to describe an ordinated matrix (also referred as morpho/eco/niche/etc.-space).
    \item The package is based on a highly modular architecture were metrics for measuring disparity and tests for testing evolutionary hypothesis are provided as standalone functions.
    \item This modular architecture allows great plasticity in using this package, namely by letting users define their own metric of disparity.
    \item The package also provides numerous tools for modifying, summarising and plotting \texttt{dispRity} objects.
\end{enumerate}

\end{abstract}

\noindent (Keywords: disparity, ordination, multidimensionality, disparity through time, palaeobiology, ecology)\\

\vspace{1.5in}

\newpage 

%---------------------------------------------
%
%       INTRODUCTION
%
%---------------------------------------------

\section{Introduction}

Multidimensionality is a major aspect of natural sciences were variables co-vary through time and space from the organism to the environmental level.

In palaeobiology multidimensionality is a popular way to study morphological diversity and comes as a interesting alternative to unidimensional taxonomic diversity analysis \citep{Hopkins2017}.


Although the concept of morphological disparity is well defined since more than two decades \citep[e.g.][]{gould1991disparity,Foote01071994,Foote29111996}, its specific definition varies widely among authors.
In fact disparity Disparity is by nature multidimensional \citep{lloyd2016estimating} and thus there are many ways to measure it.



Hereafter we will make a clear distinction between the multidimensional space which is a specific mathematical object and the disparity, which is a metric describing or summarising one or more aspects of this space.
Both can be defined in various ways depending on the authors, for example, disparity is defined as the weighted mean pairwise dissimilarity in \cite{Close2015} or the ellipsoid volume in \cite{DonohueDim} and the multidimensional space is defined as the morphospace in \cite{raup1966geometric} and the morpho-functional space in \cite{diaz2016global}.

The multidimensional space can be defined in many ways and arise from many mathematical transformations of the data such as the pairwise distance matrix \citep{Close2015}, a principal coordinates analysis \citep[PCO;][]{Brusatte12092008}, a principal components analysis \citep[PCA;][]{zelditch2012geometric}, a multidimensional scaling \citep[MDS;][]{DonohueDim}, etc.
Similarly, disparity metrics \citep[or indices;][]{Hopkins2017} can defined in many ways \citep[e.g.][or combinations thereof]{Wills2001,Ciampaglio2001,foth2012different,DonohueDim,Hughes20082013,finlay2015morphological,Close2015,diaz2016global}.
Finally, difference between disparity metrics can also be measured in many ways: using NPMANOVA \citep[e.g.][]{Brusatte12092008}, multidimensional permutation test \citep[e.g.][]{diaz2016global} or even simple confidence interval overlap \citep[e.g.][]{halliday2016eutherian}.

This variety of definitions and analysis have been developed in an equal variety of softwares such as \texttt{GINGKO} in javascript \citep{bouxin2005ginkgo,de2007ginkgo} or \texttt{geomorph} \citep{adams2013geomorph,adams2017geometric}, \texttt{Claddis} \citep{Claddis}, or \texttt{vegan} \citep{oksanen2007vegan} in R \citep{R}.
This results in the need to learn different languages (or at least - when restricted to R - different packages with different standards) as well as making analysis sometimes idiosyncratic and often complex to repeat since they are based on a particular feature from a particular software.
For example, in the excellent and widely used \texttt{geomorph} package morphological disparity analysis can be ran using the \texttt{morphol.disparity} function.
Unfortunately, however, the multidimensional space can only be defined as the ordination of the procrustes transform of geometric morphometric landmarks, the disparity can only be defined as the Procrustes variance and the difference between groups can only be measured through permutation tests \citep{zelditch2012geometric,adams2013geomorph,adams2017geometric}.

This package, is the first, to our knowledge, to be entirely dedicate to multidimensional analysis.
It has a highly modular architecture and allow users to define their own multidimensional space, use their own definition of the disparity metric and measure the differences between metrics in their own terms.
Furthermore, both a specific design and set of functions (through the \texttt{dispRity} object) allows to design analysis into easy and repeatable pipelines.

\section{Description}

\subsubsection{The package's data}

Beck and Lee data
Beck and Lee tree
Beck and Lee ages

disparity  disparity data

\subsubsection{Running example}


\subsection{The \texttt{dispRity} package}

\subsubsection{The package's main functions}

\begin{table}
    \begin{tabular}{ll}
        \hline
        Function & Description \\ 
        \hline
        \texttt{geomorph.ordination} & Imports data from the \texttt{geomorph} package. \\
        \texttt{Claddis.ordination} & Imports data from the \texttt{Claddis} package. \\
        \texttt{custom.subsamples} & Separates ordinated data in custom subsamples. \\
        \texttt{time.subsamples} & Separates ordinated data in time subsamples. \\
        \texttt{boot.matrix} & Bootstraps and rarefies ordinated data or a \texttt{dispRity} object. \\
        \texttt{dispRity} & Calculates disparity from an ordinated matrix or a \texttt{dispRity} object. \\
        \texttt{summary.dispRity} & Summarises a \texttt{dispRity} object. \\
        \texttt{plot.dispRity} & Plots a \texttt{dispRity} object. \\
        \texttt{test.dispRity} & Applies statistical tests to a \texttt{dispRity} object.\\
        \texttt{dispRity.per.group} & Pipelined disparity per groups analysis. \\
        \texttt{dispRity.through.time} & Pipelined disparity through time analysis. \\
        \texttt{dispRity.metrics} & A set of implemented disparity metrics functions.\\
        \texttt{dispRity.utilities} & A set of functions for modifying a \texttt{dispRity} object.\\
        \hline
    \end{tabular}
    \caption{Main functions in the \texttt{dispRity} package.}
\end{table}

\subsection{Fork flow}

Workflow figure

\subsubsection{Input data}

\subsubsection{Manipulating the data}

\subsubsection{Measuring disparity}

\subsubsection{Summarising the data}

Link to graph one two three

\subsubsection{Testing hypothesis}

\subsection{The \texttt{dispRity} object}

Disparity object structure figure

\subsection{Additional functions and data}

The package also contains additional functions not directly linked to the disparity workflow nor to the \texttt{dispRity} object but that can be used as utilities for the studying discrete morphological data.

\begin{table}
    \begin{tabular}{ll}
        \hline
        Function & Description \\ 
        \hline
        \texttt{char.diff} & Calculates character differences. \\
        \texttt{clean.data} & Cleans phylogenetic data (tree and table). \\
        \texttt{sim.morpho} & Simulates morphological discrete data. \\
        \texttt{slice.tree} & Slice through a phylogenetic tree. \\
        \texttt{space.maker} & Creates multidimensional spaces. \\
        \texttt{tree.age} & Calculates the age of nodes and tips in a tree. \\
        \hline
    \end{tabular}
    \caption{Additional functions in the \texttt{dispRity} package.}
\end{table}

Additionally, this package also contains an ecology type data set from McClean

McClean data

\section{Conclusion}

This package is good.

\section{Data availability and reproducibility}
Data are available on Dryad or Figshare.
Code for reproducing the analyses is available on GitHub (\url{github.com/TGuillerme/SpatioTemporal_Disparity}).

\section{Acknowledgments}
Thanks to Dave Bapst, Martin Brazeau, Natalie Cooper, Andrew Jackson, Graeme Lloyd and Emma Sherratt.
I acknowledge support from European Research Council under the European Union's Seventh Framework Programme (FP/2007 – 2013)/ERC Grant Agreement number 311092 awarded to Martin D. Brazeau.

\section{Funding}
This work was funded by the European Research Council under the European Union's Seventh Framework Programme (FP/2007–2013)/ERC Grant Agreement number 311092.

\bibliographystyle{sysbio}
\bibliography{References}

\end{document}
